%%\VignetteIndexEntry{entry name for CRAN}
%%\VignetteDepends{texor,rebib}

\begin{abstract}
  Many of the early R packages used Sweave to write vigneettes,
  which will be shown to users as pdf format on CRAN. However, as
  time goes on, R Markdown, a lightweight markup language, has
  begun to gradually replace Sweave and better present content
  on CRAN in the form of HTML.

  In order to help many R package developers who haven't used R
  Markdown or don't have time to do the format conversion manually
  to migrate from Sweave to R Markdown, \code{texor} helps people
  automate this conversion process.
\end{abstract}


\title{A Simple Example for rnw_to_rmd Function}
\author{Yinxiang Huang\\Beihang University\\\email{21375146@buaa.edu.cn}}

\section{Introduction to Sweave and Knitr}

Sweave options will be automatically converted to knitr.

\Rcodeplaceholder{}

To remove warning messages, setup code chunk use knitr option \texttt{warning=FALSE}.

\Rcodeplaceholder{}

When the option OutDec is not \texttt{.}, put numbers in \texttt{\textbackslash{}text}.

\Rcodeplaceholder{}

This is the first test. abc \Sexpr{0.6} def

another test $a = \Sexpr{0.6}$.

and the last one $a = \Sexpr{'0.6'}$.

\Rcodeplaceholder{}

This is the second test. abc \Sexpr{0.6} def

another test $a = \Sexpr{0.6}$.

and the last one $a = \Sexpr{'0.6'}$.

Inline R code in verbatim environment will also be converted.

\begin{verbatim}
try this:
\Sexpr{1+1}
\end{verbatim}

\section{Generate Image from R Code Chunk}

The first image:

\Rcodeplaceholder{}

Insert a reference to the first image: \ref{fig:test}. And for bibtex\cite{Yinxiang:2024}.

Let's test another image from R code chunk:

\Rcodeplaceholder{}

Show R logo from file:

\begin{figure}
\centering
\includegraphics[width=0.5\textwidth]{Rlogo.png}
\caption{R logo}
\end{figure}

\section*{References}
\bibliography{example}

